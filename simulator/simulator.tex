\documentclass[a4paper,12pt]{article}
\usepackage{graphicx}
\usepackage{hyperref}
\usepackage{subfig}
\usepackage[textwidth=16cm,textheight=24cm]{geometry}
\begin{document}

\begin{center}
\vspace{1cm}
\LARGE{\textbf{DEPARTMENT OF COMPUTER SCIENCE AND ENGINEERING}}\\
\vspace{1cm}
\Large{IIT GUWAHATI}\\

\vspace{5 mm}
\begin{figure}[h]
\centering
\includegraphics[scale=1]{iitglogo.jpg}
\end{figure}
\vspace{5 mm}
\textbf{CS244: SYSTEM PROGRAMMING}\\
\vspace{2.5cm}
\textbf{Process Scheduler}\\


\vspace{1.5cm}
\normalsize{\textit{By:}}

\vspace{5mm}
\large

Anirudh 120101064 $\vert$ Sathwik 120101051 $\vert$ Roshan 120101062 \\



\end{center}
\vspace{15mm}
\clearpage
\begin{LARGE}
\begin{center}
\textbf{ACKNOWLEDGEMENTS}

\end{center}
\end{LARGE}
\vspace{5mm}
We extend our profound gratitude to our project guides Mr Arnab Sarkar and Mr Santosh Biswas, for their interest, guidance and suggestions throughout the course of the project.We feel honoured and privileged to work under them.They shared their vast pool of knowledge with us that helped us steer through all the difficulties with ease.This project would not have been possible without their guidance and we would like to thank them for everything they have done for us.
\clearpage
\tableofcontents
\clearpage

\section*{Introduction}
In this project we tried to make \emph{Primitive Process Scheduler} using Python as our language and Gnuplot to draw graphs. We also made a \emph{Round-Robin Scheduler} for comparision with \emph{Primitive Process Scheduler}.\\ In the graps we plotted two types of graphs:
\begin{itemize}
\item Misses vs Number of processors
\item Misses vs Number of tasks.
\end{itemize}

\section*{Softwares Requirements}
We need Python 2.7, Gnuplot, Gnuplot.py, numpy preinstalled before running the code. Gnuplot is used to make 2-D and 3-D graps where as Gnuplot.py interfaces Gnuplot with python. Numpy gives addition mathematical capabilities to Python.

\section*{Inputs and Outputs}
Inputs are given in a \textit{generate.txt} file. The format of input is as follows:
\begin{itemize}
\item $1^{st}$ line contains the number of test cases.
\item $2^{nd}$ line contains number of processors.
\item $3^{rd}$ line contains number of tasks.
\item $4^{th}$ line contains processing time.


\end{itemize}
\section*{Logic and Code Explanation}
\subsection*{Round Robin Processor}
\subsection*{Primitive Process Scheduler}

\section*{Possible Improvements}



\end{document}
